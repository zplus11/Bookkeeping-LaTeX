% This is illustration.tex.tex printed using https://github.com/zplus11/Bookkeeping-LaTeX.git

\documentclass[a4paper, 10pt]{article}

\usepackage{geometry}
\usepackage{tabularx}
\usepackage{booktabs}
\usepackage{enumerate, enumitem}
\usepackage{titlesec}
\usepackage{array}
\usepackage{makecell}

\geometry{margin=1cm}
\setlength{\parindent}{0pt}
\pagestyle{empty}

\titleformat{\section}{\bfseries\large}{\thesection. }{0cm}{}

\newcommand{\insertitem}[1]{\item #1}
\NewDocumentCommand{\mylist}{>{\SplitList{;;;}}m}
{
	\begin{enumerate}[itemsep = 0pt, topsep = 0pt, parsep = 0pt, partopsep = 0pt, leftmargin = *]
		\ProcessList{#1}{\insertitem}	
	\end{enumerate}
}

\newcommand{\journal}[1]{
	\begin{center}
		{\bfseries Journal}
	\end{center}
	\vspace{-.3cm}
	\begin{tabularx}{\linewidth}{|c|X|c|r|r|}
		\hline
		\textbf{Date} & \textbf{Particulars} & \textbf{LF} & \textbf{Dr Amt} & \textbf{Cr Amt} \\
		\hline
		#1
		\hline
	\end{tabularx}
}

\newcommand{\ledger}[3]{
	\begin{center}
		{Dr. \hfill {\bfseries #1} \hfill Cr.}
	\end{center}
	\vspace{-.3cm}
	\noindent
	\begin{minipage}{.5\linewidth}
		\begin{tabularx}{\linewidth}{|c|X|c|r|}
			\hline
			\textbf{Date} & \textbf{Particulars} & \textbf{JF} & \textbf{Amt} \\
			\hline
			#2
			\hline
		\end{tabularx}
	\end{minipage}
	\hspace{-.2cm}
	\begin{minipage}{.5\linewidth}
		\begin{tabularx}{\linewidth}{|c|X|c|r|}
			\hline
			\textbf{Date} & \textbf{Particulars} & \textbf{JF} & \textbf{Amt} \\
			\hline
			#3
			\hline
		\end{tabularx}
	\end{minipage}
}		

\newcommand{\jyear}[1]{{\bfseries #1} & & & & \\}
\newcommand{\jdr}[4]{#1 & #2 a/c \hfill ...Dr \hfill & #3 & #4 & \\}
\newcommand{\jcr}[3]{& \hspace{.5cm} To #1 a/c & #2 & & #3 \\}
\newcommand{\jnar}[1]{& {(\itshape #1)} & & & \\ \cline{2-2}}

\newcommand{\lyear}[1]{{\bfseries #1} & & & \\}
\newcommand{\ldr}[4]{#1 & To #2 a/c & #4 & #3 \\}
\newcommand{\lcr}[4]{#1 & By #2 a/c & #4 & #3 \\}
\newcommand{\drbal}[2]{#1 & To balance c/d & & #2 \\}
\newcommand{\crbal}[2]{#1 & By balance c/d & & #2 \\}
\newcommand{\total}[1]{\cline{4-4} & & & #1 \\}
\newcommand{\mt}{& & & \\}
\title{Bookkeeping with \LaTeX}
\date{\today}

\begin{document}
	\maketitle

	%%%%%%%%%%%%%%%%%%%%%%
	% JOURNAL ENTRIES
	%%%%%%%%%%%%%%%%%%%%%%

	\journal{That Company Pvt. Ltd.}{2024}{

		\jyear{2023}

		\jdr{29 Dec}{Cash}{1}{500000}
		\jcr{Capital}{2}{500000}
		\jnar{being capital introduced}

		\jdr{29 Dec}{Bank}{3}{200000}
		\jcr{Cash}{4}{200000}
		\jnar{being cash deposited into bank}

		\jdr{30 Dec}{Furniture}{}{20000}
		\jdr{}{Computers}{}{10000}
		\jdr{}{Purchases}{}{20000}
		\jcr{Bank}{}{50000}
		\jnar{being furniture, computers and stock purchased}

		\jyear{2024}

		\jdr{1 Jan}{Cash}{9}{30000}
		\jdr{}{Bank}{10}{120000}
		\jcr{Sales}{11}{150000}
		\jnar{being sales made to bank and cash}

	}

	%%%%%%%%%%%%%%%%%%%%%%
	% LEDGER POSTING
	%%%%%%%%%%%%%%%%%%%%%%

	Ledger posting: W.I.P.

	% COMPUTERS ACCOUNT 
	\ledger{Computers a/c}{
		\ldr{30 Dec}{Bank}{10000}{}
		\mt
}{
		\mt		\mt
}

	% CASH ACCOUNT 
	\ledger{Cash a/c}{
		\ldr{29 Dec}{Capital}{500000}{1}
		\ldr{1 Jan}{Sales}{30000}{9}
		\mt
}{
		\lcr{29 Dec}{Bank}{200000}{4}
		\mt		\mt
}

	% FURNITURE ACCOUNT 
	\ledger{Furniture a/c}{
		\ldr{30 Dec}{Bank}{20000}{}
		\mt
}{
		\mt		\mt
}

	% BANK ACCOUNT 
	\ledger{Bank a/c}{
		\ldr{29 Dec}{Cash}{200000}{3}
		\ldr{1 Jan}{Sales}{120000}{9}
		\mt
}{
		\lcr{30 Dec}{Furniture}{50000}{}
		\mt		\mt
}

	% PURCHASES ACCOUNT 
	\ledger{Purchases a/c}{
		\ldr{30 Dec}{Bank}{20000}{}
		\mt
}{
		\mt		\mt
}

	% CAPITAL ACCOUNT 
	\ledger{Capital a/c}{
		\mt		\mt
}{
		\lcr{29 Dec}{Cash}{500000}{2}
		\mt
}

	% SALES ACCOUNT 
	\ledger{Sales a/c}{
		\mt		\mt
}{
		\lcr{1 Jan}{Cash}{150000}{11}
		\mt
}

\end{document}
% End of file.